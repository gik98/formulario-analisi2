\documentclass[a4paper,portrait,columns=3,5pt]{cheatsheet}
\title{Formulario Analisi 2}
%\author{Elena Acinapura, Giacomo Fabris}
\author{}
\date{\today}

\usepackage{commath}
\usepackage{graphicx}
\usepackage{mathtools}
\usepackage{commath}
\usepackage{amssymb}
\usepackage{amsmath}
\usepackage{bigints}
\usepackage[utf8]{inputenc}

\begin{document}
%\maketitle

\section{Formule trigonometriche}
\subsection{Addizione e sottrazione}
$$\sin (\alpha + \beta) = \sin(\alpha) \cos(\beta) + \cos(\alpha)\sin(\beta)$$
$$\sin(\alpha - \beta) = \sin(\alpha) \cos(\beta) - \cos(\alpha)\sin(\beta)$$
$$\cos(\alpha + \beta) = \cos(\alpha) \cos(\beta) - \sin(\alpha) \sin(\beta)$$
$$\cos(\alpha - \beta) = \cos(\alpha) \cos(\beta) + \sin(\alpha) \sin(\beta)$$
$$\tan(\alpha + \beta) = \frac{\tan(\alpha) + \tan(\beta)}{1 - \tan(\alpha)\tan(\beta)}$$
$$\tan(\alpha - \beta) = \frac{\tan(\alpha) - \tan(\beta)}{1 + \tan(\alpha)\tan(\beta)}$$
\subsection{Duplicazione}
$$\sin(2\alpha) = 2\sin(\alpha)\cos(\alpha)$$
$$\cos(2\alpha) = \cos^2 (\alpha) - \sin^2 (\alpha)$$
$$\tan(2\alpha) = \frac{\tan(\alpha)}{1 - \tan^2(\alpha)}$$
$$\cot(2\alpha) = \frac{\cot^2(\alpha) - 1}{2\cot(\alpha)}$$
\subsection{Bisezione}
$$\sin\left(\frac{\alpha}{2}\right) = \pm \sqrt{\frac{1 - \cos(\alpha)}{2}}$$
$$\cos\left(\frac{\alpha}{2}\right) = \pm \sqrt{\frac{1 + \cos(\alpha)}{2}}$$
$$\tan\left(\frac{\alpha}{2}\right) = {\frac{1 - \cos(\alpha)}{\sin(\alpha)}}$$
$$\cot\left(\frac{\alpha}{2}\right) = {\frac{1 + \cos(\alpha)}{\sin(\alpha)}}$$
\subsection{Werner}
$$\sin(\alpha)\sin(\beta) = \frac{1}{2} \left[\cos(\alpha - \beta) - \cos(\alpha + \beta)\right]$$
$$\cos(\alpha)\cos(\beta) = \frac{1}{2} \left[\cos(\alpha - \beta) + \cos(\alpha + \beta)\right]$$
$$\sin(\alpha)\cos(\beta) = \frac{1}{2} \left[\sin(\alpha - \beta) + \sin(\alpha + \beta)\right]$$

\section{Derivate fondamentali}
$$ (\tan(x))' = 1 + \tan^2(x) = \sec^2(x)$$
$$ (\arcsin(x))' = \frac{1}{\sqrt{1 - x^2}}$$
$$ (\arccos(x))' = - \frac{1}{\sqrt{1 - x^2}}$$
$$ (\arctan(x))' = \frac{1}{1 + x^2}$$
$$ (\sec(x))' = \tan(x)\sec(x)$$
$$ (\csc(x))' = -\cot(x)\csc(x)$$
$$ (\cot^n(x))' = - n \cdot \csc(x) \cdot \sec(x) \cdot \cot^n (x)$$
$$ (a^x)' = a^x \ln(a)$$
$$ (\log_a~x )' = \frac{1}{x\ln(a)}$$

\section{Roba iperbolica}
$$ \sinh(x) = \frac{e^x - e^{-x}}{2} $$
$$\cosh(x) = \frac{e^x + e^{-x}}{2}$$
$$ \cosh^2(x) - \sinh^2(x) = 1 $$
$$ (\cosh(x))' = \sinh(x)$$
$$ \quad (\sinh(x))' = \cosh(x) $$
\section{Sostituzioni notevoli}
Per funzioni razionali in seno e coseno: 
$$t \coloneqq \tan(\frac{x}{2})\quad \dif x = \frac{2}{1 + t^2} \dif t$$
$$ \sin(x) = \frac{2t}{1+t^2} \quad \cos(x) = \frac{\-t^2}{1+t^2}$$

Per radici del tipo $\sqrt{a - x^2}$:
$$ x = \sqrt{a} \sin(t) $$

\section{Disuguaglianze utili}
$$ \abs{xy} \leq \frac{1}{2}\left(x^2 + y^2\right) $$
$$ \abs{\sin(xy)} \leq \abs{xy}\quad \text{per xy$\rightarrow$ 0}$$
\section{Sviluppi in serie}
$$ e^x = 1 + x + \frac{x^2}{2!} + \frac{x^3}{3!} + o(x^3)$$
$$ \ln(1 + x) = x - \frac{x^2}{2} + \frac{x^3}{3} + o(x^3)$$
$$ \sin(x) = x - \frac{x^3}{3!} + \frac{x^5}{5!} + o(x^6)$$
$$ \cos(x) = 1 - \frac{x^2}{2!} + \frac{x^4}{4!} + o(x^5)$$
$$ \tan(x) = x + \frac{x^3}{3} + \frac{2x^5}{15} + o(x^6)$$
$$ \arctan(x) = x - \frac{x^3}{3} + \frac{x^5}{5} + o(x^6)$$
$$ \frac{1}{1-x} = 1 + x + x^2 + x^3 + o(x^3)$$
$$(1 + x) ^ \alpha = 1 + \alpha x + \frac{\alpha (\alpha - 1)x^2}{2!} + \dots $$
$$ + \frac{\alpha(\alpha - 1)\dots (\alpha - n - 1) x^n}{n!} + o(x^n) $$

\section{Integrali utili}
$$ \int \sin^2(x) \dif x = \frac{1}{2} (x - \sin(x) \cos(x)) $$
$$ \int \sin^3(x) \dif x = \frac{1}{12} (\cos(3x)-9\cos(x)) $$
$$ \int \sin^4(x) \dif x = \frac{1}{32} (12x - 8\sin(2x)+\sin(4x)) $$
$$ \int \sin^n(x) \dif x = -\frac{1}{n} \cos(x) \sin^{n-1}(x) + $$ 
$$ + \frac{n-1}{n} \int \sin^{n-2}(x) \dif x $$
$$ \int \cos^2(x) \dif x = \frac{1}{2} (x + \sin(x)\cos(x)) $$
$$ \int \cos^3(x) \dif x = \frac{1}{12} (\sin(3x) + 9\sin(x))$$
$$ \int \cos^4(x) \dif x = \frac{1}{32} (12 x + 8\sin(2x)+\sin(4x))$$
$$ \int \cos^n(x) \dif x = \frac{1}{n} \sin(x) \cos^{n-1}(x) + $$
$$ + \frac{n-1}{n} \int \cos^{n-2}(x) \dif x $$
$$ \int \sin(x) ~ \cos(x) \dif x = -\frac{1}{2} \cos^2(x) $$ 
$$ \int \tan(x) = -\ln(\cos(x)) $$
$$ \int \sec(x) \dif x = \ln (\tan(x) + \sec(x)) $$
$$ \int \sec^2(x) \dif x = \tan(x) $$
$$ \int \csc (x) \dif x = \ln \left( \tan\left(\frac{x}{2}\right)\right)$$
$$ \int \csc^2(x) \dif x = \cot(x) $$
$$ \int \arctan(x) \dif x = x \arctan(x) - \frac{1}{2} \ln (1 + x^2)$$
$$ \int \log(f(x)) \dif x = x \log (f(x)) - \int \frac{x f'(x)}{f(x)} \dif x $$
$$ \int \sqrt{a - x^2} \dif x = \frac{x}{2} \sqrt{a - x^2} + \frac{a}{2} \arcsin \left(\frac {x}{a}\right) = $$
$$ =  \frac{x}{2} \sqrt{a - x^2} + \frac{a}{2} \arctan \left(\frac{x}{\sqrt{a-x^2}}\right) $$
$$ \int \sqrt{a + x^2} \dif x =  \frac{x}{2} \sqrt{a + x^2} + $$ 
$$ + \frac{a}{2} \ln \abs{\frac{x + \sqrt{a+x^2}}{\sqrt{a}}} $$

\section{Cambiamento delle variabili}
\begin{equation*}
	\int_R f(\vec {x}) d\vec{x} \underset{{\vec{u} = \vec{g} (\vec{x})}}{=} \int_{\vec{g}^{-1}(R)} f(\vec {u}) \abs{det J_g(\vec u)} d\vec{u}
\end{equation*}

\subsection{Coordinate sferiche}
% \begin{center}
% 	\includegraphics[scale=.2]{spherical.png}
% \end{center}
\begin{equation*}
	\vec{g} :
	\begin{bmatrix}
		x \\
		y \\
		z \\
	\end{bmatrix} =
	\begin{bmatrix}
		\rho \sin \phi \cos \theta \\
		\rho \sin \phi \sin \theta \\
		\rho \cos \phi             \\
	\end{bmatrix}
\end{equation*}
\begin{equation*}
	\rho = \sqrt{x^2 + y^2 + z^2}
\end{equation*}
\begin{equation*}
	\abs{detJ_{\vec{g}}(\vec{u})} = \rho^2 \sin(\phi)
\end{equation*}
\subsection{Coordinate cilindriche}
\begin{equation*}
	\vec{g} :
	\begin{bmatrix}
		x \\
		y \\
		z \\
	\end{bmatrix} =
	\begin{bmatrix}
		r \cos \theta \\
		r \sin \theta \\
		z             \\
	\end{bmatrix}
\end{equation*}
\begin{equation*}
	r = \sqrt{x^2 + y^2}
\end{equation*}
\begin{equation*}
	\abs{detJ_{\vec{g}}(\vec{u})} = r
\end{equation*}
\section{Curve e superfici}
\textbf{Curva} $\gamma:[a,b]\rightarrow\mathbb{R}^n$\\
Per curve cartesiane per cui $y = g(x)$ , $$\norm{\gamma'(t)} = \sqrt{1 + (g'(x))^2}$$\\
Per curve in coord. polari: $x = \rho(t) \cos(\theta(t))$, $y = \rho(t) \sin(\theta(t))$
$$\norm{\gamma'(t)} = \sqrt{\rho'^2(\theta) + \rho^2\theta'^2} $$\\
$$ L(\gamma) = \int_{a}^{b} \norm{\gamma'(t)} \dif t$$\\
Per $f$ funzione a valori scalari definita sulla curva
$$ \int\limits_{\gamma} f\dif s = \int_a^b f(\gamma(t)) \norm{\gamma'(t)} \dif t$$\\
Per $F$ campo vettoriale definito sulla curva
$$ \int\limits_\gamma \omega_{F} = \int_a^b \langle F(\gamma(t)), \gamma'(t)\rangle \dif t = $$\\
$$ = \int\limits_a^b \sum_{i=1}^n\left[F_i(\gamma(t))~\gamma_i'(t)\right] \dif t$$

\textbf{Superficie} S con par. $\phi:D\subset\mathbb{R}^2\rightarrow\mathbb{R}^n$\\
Per superfici cartesiane per cui $z = g(x,y)$, $$\norm{\vec N} = \sqrt{1 + \norm{\nabla g}^2}$$
$$ Area(S) = \iint\limits_D \norm{\phi_u \wedge \phi_v} \dif u \dif v$$
Per $f$ funzione a valori scalari definita sulla superficie
$$ \iint\limits_S f \dif \sigma = \iint\limits_D f ~\norm{\phi_u \wedge \phi_v} \dif u \dif v $$
\textbf{Flusso attraverso una superficie}\\ Per $F$ campo vettoriale definita sulla superficie
$$ \Phi_S(F) = \int\limits_S \langle F, n_e \rangle \dif \sigma = $$ \\ 
$$ \iint\limits_D \langle F(\phi(u,v)), \phi_u \wedge \phi_v \rangle \dif u \dif v$$

\section{Forme geometriche}

\textbf{Paraboloidi}: $z = x^2 + y^2 + a$, vertice in $a$

\textbf{Coni}: $z = a - \sqrt{x^2 + y^2}$, vertice in $a$, verso il basso. Verso l'alto: $a + \sqrt{x^2 + y^2}$

\textbf{Ellissoidi}: $\frac{x^2}{a^2} + \frac{y^2}{b^2} + \frac{z^2}{c^2} = 1$, lunghezza del semiasse lungo $x$ è $a$, ecc. 
\section{Teoremi principali}
\begin{itemize}
	\item $f$ differenziabile in $x_0~ \Rightarrow~f$ continua in $x_0 $
	\item \textbf{Teorema del differenziale totale} \\ Derivate parziali di $f$ esistono in un intorno di $x_0$ e sono continue in $x_0~\Rightarrow ~ f$ differenziabile in $x_0$ 
	\item \textbf{Schwarz con due alternative} \\ 1) Derivate seconde miste $f_{xy}$ e $f_{yx}$ di $f$ esistono in un intorno di $x_0$ e sono continue in $x_0~\Rightarrow~f_{xy} = f_{yx}$ in $x_0$
			\\ 2) $f_{xy}$ e $f_{yx}$ esitono nell'intorno e sono differenziabili nel punto $\Rightarrow~_{xy} = f_{yx}$ in $x_0$
	\item $f : A \rightarrow \mathbb{R}^m$ continua, $A$ connesso $\Rightarrow~f(A)$ è connesso
	\item \textbf{Teorema sui punti estremali}\\
			$f:A\subset \mathbb{R}^n \rightarrow \mathbb{R}$, $x_0 \in A$ punto critico, $f\in C^2(A)$\\
			- $Hf(x_0)$ def. negativa $\Rightarrow x_0$ p. di max stretto \\
			- $Hf(x_0)$ def. positiva $\Rightarrow x_0$ p. di min stretto \\
			- $Hf(x_0)$ indef. $\Rightarrow x_0$ p. di sella\\
			- $x_0$ p. di max $\Rightarrow Hf(x_0)$ semidef. neg.\\
			- $x_0$ p. di min $\Rightarrow Hf(x_0)$ semidef. pos.
	\item \textbf{Forme diff. esatte e poteziale}\\
			$A$ aperto, $\omega\in C^1(A)$ esatta di potenziale $f$, $\gamma :[a,b]\rightarrow \mathbb{R}$ curva con supporto in A 
			$\Rightarrow \int_\gamma \omega = f(\gamma(b)) - f(\gamma(a))$
	\item \textbf{Forme differenziali esatte} \\
			$E \subset \mathbb{R}^n$ aperto connesso. $\omega$ forma differenziale continua in $E$ associata al campo $F$ è \textbf{esatta in E} se esiste la funzione potenziale $U: E \to \mathbb{R}$ di classe $C^1$ tale che $\nabla U = F$.
			
	\item \textbf{Sulle forme differenziali esatte}\\
			$\omega \in C^1(A)$, $A$ aperto e connesso; sono equivalenti:\\
				- $\omega$ è esatta in A\\
				- per ogni $\gamma_1$ e $\gamma_2$ parametrizzazioni della stessa curva contenuta in A, 
					$\int\limits_{\gamma_1} \omega = \int\limits_{\gamma_2}\omega$\\
				- per ogni $\gamma$ curva chiusa in A, $\int\limits_\gamma \omega = 0$
	\item \textbf{Invarianza omotopica}\\
		$E \in \mathbb{R}^n$ aperto connesso e $\omega$ \textbf{chiusa} in $E$. Se $\gamma^{(0)}$ e $\gamma^{(1)}$ sono due curve omotope in $E$, allora $\int\limits_{\gamma^{(0)}} \omega$ = $\int\limits_{\gamma^{(1)}} \omega$ \\
		$\to$ una forma definita in un insieme non semplicemente connesso a causa di un buco \textbf{e la forma è chiusa}, posso calcolare l'integrale della forma su una curva che circonda il buco: se vale 0, per l'invarianza omotopica è 0 l'integrale di qualsiasi curva attorno al buco, dunque è esatta.
	
	\item \textbf{Equivalenza per l'esattezza}\\
			$A\subset \mathbb{R}^n$, $\omega \in C^1(A)$\\ $A$ semplicemente connesso e $\omega$ chiusa $\to \omega$ esatta in A
	\item \textbf{Formule di Green}\\
			$D \subset \mathbb{R}^2$ dominio regolare e normale a tratti, $f \in C^1(D)$ a valori scalari, valgono:\\
			 $$\iint\limits_D\dfrac{\partial f}{\partial x} \dif A = \int_{+\partial D} f \dif y$$\\
			$$\iint\limits_D \dfrac{\partial f}{\partial y} \dif A = -\int_{+\partial D} f \dif x$$\\
			$$ Area(D) = \frac{1}{2} \int\limits_{+\partial D} x\dif y - y\dif x = $$ \\
			 $$ \underset{polari}{=} \frac{1}{2}\int\limits_{\theta_1}^{\theta_2} \rho^2(\theta) \dif \theta$$
	\item \textbf{Teorema di Stokes}\\ 
			$S$ superficie regolare bordo con parametrizzazione $\phi$, $F \in C^1$ sulla curva 
			$$ \int\limits_S \langle rot(F), n_e \rangle \dif \sigma = \int\limits_{+\partial S}\langle F, T \rangle \dif s$$ 
	\item \textbf{Teorema della divergenza}\\
			$T$ dominio regolare chiuso, $F \in C^1$ su T
			$$ \iiint\limits_T div(F) \dif V = \iint\limits_{+\partial T} \langle F, n_e \rangle \dif \sigma =$$ \\
			$$ = \Phi_{+\partial T}(F)$$
\end{itemize}

\end{document}
